\chapter{Methods}


\section{Key resources}

\small

% Antibodies
\begin{tabularx}{\linewidth}{p{0.35\linewidth} p{0.35\linewidth} p{0.2\linewidth}}
    \regtable{Antibodies}

    Alexa Fluor 647 &Abcam &ab150075 \\\midrule
    TIA-1 &Abcam &ab40693 \\
\end{tabularx}

% Cell lines
\begin{tabularx}{\linewidth}{p{0.35\linewidth} p{0.35\linewidth} p{0.2\linewidth}}
    \regtable{Experimental Models: Cell lines}

    Hela 11ht &Weidenfeld et al., 2009 \cite{weidenfeld_inducible_2009} &N/A \\\midrule
    Hela 11ht + scAB-GFP + Renilla-MS2 &This study &N/A \\\midrule
    Hela 11ht + scAB-GFP + smGCN4-Renilla-MS2 &This study &N/A \\\midrule
    Hela 11ht + scAB-GFP + SunTag-Renilla-MS2 &This study &N/A \\\midrule
    Hela 11ht + TREAT &This study &N/A \\
\end{tabularx}

% Chemicals n peptides
\begin{tabularx}{\linewidth}{p{0.35\linewidth} p{0.35\linewidth} p{0.2\linewidth}}
    \regtable{Chemicals and Peptides}

    Amino-11-ddUTP &Lumiprobe &15040 \\\midrule
    Atto 647N NHS ester &Sigma-Aldrich &AD 647N-31 \\\midrule
    Bovine serum albumin &Sigma-Aldrich &A2153-50G \\\midrule
    Dextran sulfate &Sigma &D6001-50G \\\midrule
    Doxycycline &Sigma &D3891-1G \\\midrule
    FluoroBrite DMEM &Life Technologies &A1896703 \\\midrule
    Formamide (deionized) &Chemicon &S4117 \\\midrule
    Ganciclovir &Sigma-Aldrich &G2536-100MG \\\midrule
    JF585 HaloTag ligand &Grimm et al., 2015, 2017 \cite{grimm_general_2015,grimm_general_2017} &N/A \\\midrule
    Lipofectamine 2000 &Invitrogen &11668019 \\\midrule
    Opti-MEM™ &Gibco &31985070 \\\midrule
    Paraformaldehyde 20\% (aqueous) &Electron Microscopy Sciences &15713 \\\midrule
    ProLong Gold Antifade Mountant &Molecular Probes, Life Technologies &P36935 \\\midrule
    Puromycin &Invivogen &ant-pr-1 \\\midrule
    Sodium Arsenite solution &Sigma &35000-1L-R \\
\end{tabularx}

% Assays
\begin{tabularx}{\linewidth}{p{0.35\linewidth} p{0.35\linewidth} p{0.2\linewidth}}
    \regtable{Critical Commercial Assays}

    Bradford Protein Assay &Biorad &5000006 \\\midrule
    Renilla Luciferase Assay System &Promega &E2810 \\
\end{tabularx}

% Oligos
\begin{tabularx}{\linewidth}{p{0.35\linewidth} p{0.35\linewidth} p{0.2\linewidth}}
    \regtable{Oligonucleotides}

    Oligodeoxyribonucleotides and primers are listed in Appendix \ref{sec:appendA}. &N/A &N/A \\
\end{tabularx}

% Plasmids
\begin{tabularx}{\linewidth}{p{0.35\linewidth} p{0.35\linewidth} p{0.2\linewidth}}
    \regtable{Recombinant DNA}

    G3BP1-GFP-GFP &Wilbertz et al., 2019 \cite{wilbertz_single-molecule_2019} &Addgene \#119950 \\\midrule
    NLS-stdMCP-stdHalo &Voigt et al., 2017 \cite{voigt_single-molecule_2017} &Addgene \#104999 \\\midrule
    NLS-stdMCP-stdHalo-G3BP1 &This Study &N/A \\\midrule
    NLS-stdMCP-stdHalo-Rh1 &This Study &N/A \\\midrule
    pCAGGS-FLPe-IRESpuro &Beard et al., 2006 \cite{beard_efficient_2006} &Addgene \#20733 \\\midrule
    Renilla-MS2v5 &This Study &N/A \\\midrule
    scAB-GFP &Voigt et al., 2017 \cite{voigt_single-molecule_2017} &Addgene \#104998 \\\midrule
    smGCN4-Renilla-MS2v5 &This study &N/A \\\midrule
    SunTag-Renilla-MS2v5 &Wilbertz et al., 2019 \cite{wilbertz_single-molecule_2019} &Addgene \#119945 \\\midrule
    SunTag-Renilla-PP7-MS2v4 &Horvathova et al., 2017 \cite{horvathova_dynamics_2017} &N/A \\
\end{tabularx}

% Equipment
\begin{tabularx}{\linewidth}{p{0.35\linewidth} p{0.35\linewidth} p{0.2\linewidth}}
    \regtable{Critical Equipment}

    405 iBeam Smart &Toptica Photonics &N/A \\\midrule
    488 iBeam Smart &Toptica Photonics &N/A \\\midrule
    561 Cobolt Jive &Cobolt &N/A \\\midrule
    639 iBeam Smart &Toptica Photonics &N/A \\\midrule
    96-Well White Polystyrene Microplates &Costar &07-200-589 \\\midrule
    CFI Plan Apochromat Lambda 100x Oil/1.45 Objective &Nikon &N/A \\\midrule
    CSU-W1 Confocal Scanner Unit &Yokogawa &N/A \\\midrule
    Glass Coverslips &Paul Marienfeld GmbH &117580 \\\midrule
    iXon-Ultra-888 EMCCD Cameras &Andor &N/A \\\midrule
    Mithras Multimode Microplate Reader LB 940 &Berthold &38099 \\\midrule
    MS-2000 Motorized Stage &Applied Scientific Instrumentation &N/A \\\midrule
    TetraSpeck™ Fluorescent Microspheres Size Kit &Thermo Fisher Scientific &T14792 \\\midrule
    Ti2-E Eclipse Inverted Microscope &Nikon &N/A \\\midrule
    VS-Homogenizer &Visitron Systems GmbH &N/A \\
\end{tabularx}

% Software
\begin{tabularx}{\linewidth}{p{0.3\linewidth} p{0.3\linewidth} p{0.3\linewidth}}
    \regtable{Software and Algorithms}

    Affinity Designer 1.8.2 &Serif (Europe) Ltd &\url{affinity.serif.com/designer/} \\\midrule
    Benchling 2020 &Benchling &\url{benchling.com} \\\midrule
    Fiji 2.0.0-rc-69/1.52p &Schindelin et al., 2012 \cite{schindelin_fiji_2012} &\url{fiji.sc} \\\midrule
    Fluffy 0.2.2 &Eichenberger, 2020 \cite{eichenberger_fluffy_2020} &\url{github.com/BBQuercus/fluffy} \\\midrule
    KNIME 3.7.2 &Berthold et al., 2009 \cite{berthold_knime_2009} &\url{knime.com/knime-analytics-platform} \\\midrule
    PyMOL 2.3.3 &Schrodinger LLC. &\url{pymol.org} \\\midrule
    Python 3.7.4 &Python Software Foundation &\url{python.org} \\\midrule
    TrackMate v5.2.0 &Tinevez et al., 2017 \cite{tinevez_trackmate:_2017} &\url{imagej.net/TrackMate} \\\midrule
    VisiView 4.4.0 &Visitron Systems GmbH &\url{visitron.de/products/visiviewr-software} \\
\end{tabularx}
\normalsize

\section{Method details}

\textbf{Cell lines and culture details} \\
The previously described HeLa-11ht cell line \cite{weidenfeld_inducible_2009} was used for this study.
The integrated Flp-RMCE (recombinase-mediated cassette exchange) site allows for controlled, single-copy, genomic integration of a target gene.
In addition, to induce inserted target genes reversibly with doxycycline, cells are expressing a reverse tetracycline controlled transactivator (rtTA2S-M2).
Dulbecco’s Modified Eagle Medium (DMEM) containing 4.5 g/l glucose, 100 U/ml Penicillin, 100 \textmu g/ml Streptomycin, 4 mM L-Glutamine, and 10\% v/v Fetal Bovine Serum (FBS) was used to culture HeLa cells.
Cells were maintained at 37ºC with 5\% CO\textsubscript{2}.
Transient transfections were performed with Lipofectamine 2000 transfection reagent (Invitrogen) with Opti-MEM reduced serum medium (Gibco) according to manufacturer’s instructions but scaled down to 0.5 \textmu g plasmid DNA and 2 \textmu l Lipofectamine 2000 per 1 ml growth medium.
\\

\textbf{Plasmid construction} \\
To generate the G3BP1 coat protein fusion, the genomic sequence encoding the full-length G3BP1 Isoform 1 (Q13283-1) was inserted in-frame, downstream of the HaloTag sequence.
A 33 nt linker sequence was kept between HaloTag and G3BP1 to promote flexibility between the domains.
Assembly was performed by PCR amplification of G3BP1 and the MCP-Halo-linker backbone, followed by Gibson cloning \cite{gibson_enzymatic_2009}.
The resulting construct contains a constitutive UbiC promoter, SV40 NLS (nuclear localization signal), stdMCP (MS2 coat protein), HaloTag, 33 nt linker sequence, G3BP1 with a stop codon, and a  Woodchuck Hepatitis Virus Posttranscriptional Regulatory Element (WPRE) improving expression in lentiviral transfection (Figure \ref{fig:plasmid_1}).

The smGCN4 reporter sequence containing GCN4 epitopes was synthesized into a pUC57-Kan expression vector.
Assembly was performed by PCR amplification of smGCN4 and the \textit{Renilla}-MS2v5 backbone followed by Gibson cloning.
The resulting plasmid contains a Tet-CMV (cytomegalovirus) promoter followed by smGCN4, \textit{Renilla} luciferase, FKBP domain with a stop codon, MS2v5 stem-loop cassette (24x), CTE (constitutive transport element, and SV40 polyA tail (Figure \ref{fig:plasmid_2}).
\\

\textbf{Reporter cell line generation} \\
A day before selection, HeLa cells were seeded into a 6-well plate.
The targeting plasmid containing the reporter (2 \textmu g) and pCAGGS-FLPe-IRESpuro plasmid (2 \textmu g) were transfected using Lipofectamine 2000 (Invitrogen) according to manufacturers’ protocol \cite{beard_efficient_2006}.
On the next day, cells were preselected with 5 \textmu l/ml puromycin (Invivogen). Two days later, the puromycin-containing medium was removed and replaced with fresh growth medium containing 50 \textmu M ganciclovir (Sigma-Aldrich).
The selection was performed for 10-14 days to get resistant colonies that had undergone RCME.
Single-cell sorting into a 96-well plate was performed.
Clones were tested for reporter expression using luciferase assays.
\\

\textbf{\textit{Renilla} luciferase assays} \\
Reporter production was measured with Promega’s \textit{Renilla} Luciferase Assay System.
The described cell lines were seeded on 12-well plates.
The next day, plasmid DNA was transfected once cells reached approximately 70\% confluency as described above.
On the subsequent day, the expression of reporters was induced with 1 \textmu g/ml
    doxycycline (Sigma) for 3 hours unless otherwise specified.
Stress was induced by adding 1 mM/ml sodium arsenite (Sigma) for the last 1 hour of induction.
To measure recovery, cells were washed twice with PBS and replaced with fresh culturing medium.
At the specified time points, cells were washed once with PBS and lysed with 250 \textmu l Passive Lysis Buffer per well.
To ensure full cellular lysis, plates were gently shaken for 15 minutes at room temperature.
30 \textmu l lysate was transferred to 96-well EIA/RIA Plate (Costar) in triplicates.
Bioluminescence was measured 2 seconds after injecting 12 \textmu l \textit{Renilla} Luciferase Assay Reagent per well.
Measurements were performed on the Mithras Multimode Microplate Reader LB 940 (Berthold).
Data were normalized by protein concentration measured by standard Bradford Protein Assay (Biorad).
Means of at least 3 biological replicates (each with 3 described technical replicates) were calculated.
\\

\textbf{Single-molecule fluorescence in situ hybridization (FISH)} \\
Single-molecule RNA detection against the \textit{Renilla} coding sequence and MS2 (v5) was performed using Stellaris FISH probes (Biosearch Technologies).
Probes targeting MS2 (v4) were made by enzymatic oligonucleotide labeling \cite{gaspar_rna_2018} with Amino-11-ddUTP (Lumiprobe) and Atto647-NHS (ATTO-TEC). 

Cells were seeded on glass coverslips (Paul Marienfeld GmbH) placed in 12-well plates.
The next day, cells were transfected as indicated.
On the subsequent day, cells were treated as indicated, followed by two washes with PBS and fixation in 4\% paraformaldehyde (Electron Microscopy Sciences) diluted in PBS for 5 minutes.
Cells were washed thrice with PBS and permeabilized with 0.5\% Triton X-100 diluted in PBS for 5 minutes (if immunofluorescence was performed) or overnight at 4ºC in 70\% Ethanol.
Cells were washed two more times with PBS and prehybridized with wash buffer (2x SSC (Invitrogen), 10\% v/v formamide (Abcam)) twice for 5 minutes.
Coverslips were then incubated between 4 and 16 hours with hybridization solution (2x SSC, 10\% v/v formamide, 10\% v/v dextran sulfate, 0.5\% v/v BSA, 200 nM FISH probes) at 37ºC in humidified chambers.
Cells were washed once more with wash buffer for 30 minutes and twice with PBS.
Immunofluorescence staining was performed according to the "Immunofluorescence" section from BSA blocking to mounting.
Samples were then mounted on ProLong Gold Antifade Mountant with DAPI.
Imaging was performed as described in the “Live-cell imaging” section using sequential, single-camera acquisition.
\\

\textbf{Immunofluorescence} \\
Cells were seeded on glass coverslips (Paul Marienfeld GmbH) placed in 12-well plates.
The next day, cells were transfected as indicated.
On the subsequent day, cells were treated as indicated, followed by two washes with PBS and fixation in 4\% v/v paraformaldehyde (Electron Microscopy Sciences) diluted in PBS for 5 minutes.
Cells were washed thrice with PBS and permeabilized with 0.5\% v/v Triton X-100 diluted in PBS for 5 minutes.
Cells were washed two more times in PBS and incubated in 3\% w/v bovine serum albumin (BSA) (Sigma-Aldrich) diluted in PBS for 1 hour.
Primary antibodies (diluted in 1\% w/v BSA in PBS) were incubated for 1 hour at room temperature or overnight at 4ºC.
After washing three times with PBS, the cells were incubated with secondary antibodies (diluted in 1\% w/v BSA in PBS) for 1 hour at room temperature.
Cells were washed twice with PBS and mounted on glass slides in ProLong Gold Antifade Mountant with DAPI.
Imaging was performed as described in the “Live-cell imaging” section using sequential, single-camera acquisition.
\\

\textbf{Live-cell imaging} \\
Cells were seeded on 35 mm glass-bottom \textmu -Dish (ibidi GmbH).
The next day, cells were transfected as indicated.
On the subsequent day, cells were treated as indicated.
During the last 15 minutes of treatment, the medium was supplemented with JF585 HaloTag ligand, obtained from L. Lavis (Janelia Research Campus) \cite{grimm_general_2015,grimm_general_2017}, at 100 nM final concentration.
After incubation, cells were washed once with PBS and kept in FluoroBrite DMEM (Life Technologies) containing 10\% v/v FBS and 4 mM L-glutamine.
Cells were imaged within 15 minutes of medium exchange on an inverted Ti2-E Eclipse microscope (Nikon) with a CSU-W1 Confocal Scanner Unit (Yokogawa), two back-illuminated EMCCD cameras iXon-Ultra-888 (Andor), an MS-2000 motorized stage (Applied Scientific Instrumentation), and VisiView imaging software (Visitron Systems GmbH).
Specimens were illuminated with 561 Cobolt Jive (Cobolt), 405 iBeam Smart, 488 iBeam Smart, and 639 iBeam Smart lasers (Toptica Photonics) and a VS-Homogenizer (Visitron Systems GmbH).
All images were acquired with a CFI Plan Apochromat Lambda 100X Oil/1.45 objective (Nikon).
This setup results in a pixel size of 130 nm.
Unless otherwise indicated, excitation was performed with a 50 ms exposure time in a single plane.
The second camera was used to detect the 488 nm channel.
To ensure proper camera alignment, TetraSpeck™ Fluorescent Microspheres Size Kit (Thermo Fisher Scientific) was used to image 0.5 \textmu m fluorescent beads after each imaging session.
Cells were maintained at 37ºC and 5\% CO\textsubscript{2} within an incubation box.
\\


\section{Quantification and statistical analysis}

\textbf{Detection of mRNA spots from smFISH and colocalization} \\
Using custom-built python scripts, images were registered and maximum intensity projected in all channels.
Cells were segmented in two steps.
First, a sequential workflow comprising Gaussian filtering steps, Otsu thresholding \cite{otsu_threshold_1979}, and distance transform calculations of the DAPI channel (405 nm) was used to segment individual nuclei.
Subsequently, the nuclei were used as seeds to perform a watershed segmentation into a previously thresholded cytoplasmic map.
Cytoplasmic channels were chosen based on their signal uniformity (561 nm \textit{Renilla} or MS2v5 probe channel).
SGs were segmented in Fluffy \cite{eichenberger_fluffy_2020}.
mRNA spots were detected with a Laplacian of Gaussian filter.
Thresholds for cellular segmentation and spot detection were kept constant throughout a dataset.
To colocalize mRNA spots, Euclidean distances were calculated between spots across both channels.
To ensure proper channel-to-channel association of mRNA spots, cells with a high spot density were filtered out.
Two mRNAs were considered as colocalized when their coordinates are less than 2 px (240 nm) apart.
\\

\textbf{Processing of live-cell imaging data} \\
Images of the fluorescent beads were used to perform channel alignment.
Using the pystackreg python package \cite{thevenaz_pyramid_1998} an affine transformation was registered.
This model was subsequently re-applied to all images acquired with the secondary camera.
Fiji \cite{schindelin_fiji_2012} was used to create representative movies and images via cropping, brightness adjustments, channel merging, and scale bar annotation.
TrackMate \cite{tinevez_trackmate:_2017} was used to display tracked particles.
\\

\textbf{Quantification of translational status} \\
Tracking-based quantification of translation was performed using a KNIME \cite{berthold_knime_2009} workflow described
    in Mateju et al. \cite{mateju_single-molecule_2020} in section “Track-based analysis of translational status and colocalization” without including the described SG association component.
\\

\textbf{Statistics} \\
Results are presented as the mean $\pm$ confidence interval (95\%) of independent experiments.
Significant differences between variables are based on independent sample t-tests.
P-values are indicated using stars in each figure.
Each star corresponds to the following p-values:
\begin{itemize}
    \itemsep0em 
    \item ns: 5e\textsuperscript{-2} < p <= 1
    \item *: e\textsuperscript{-2} < p <= 5e\textsuperscript{-2}
    \item **: e\textsuperscript{-3} < p <= e\textsuperscript{-2}
    \item ***: e\textsuperscript{-4} < p <= e\textsuperscript{-3}
    \item ****: p <= e\textsuperscript{-4}
\end{itemize}

\section{Data and code availability}
The code used to analyze and visualize the data together with plasmid maps is available on GitHub\footnote{\href{https://github.com/BBQuercus/master-thesis}{https://github.com/BBQuercus/master-thesis}}.
All data supporting the findings of this study are available on request.