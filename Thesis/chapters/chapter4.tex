\chapter{Discussion}


The role of many SG-proteins in unstressed cells is still unclear.
An understanding of these proteins, such as G3BP1 could provide important insights into how SGs form and what functional purpose they serve.
This study approaches this mystery experimentally.

Fusing G3BP1 to MCP has profound effects on their associated reporter transcripts.
First, mRNAs are more likely to get recruited to SGs.
The nature of transient transfections used for these sets of experiments typically does not achieve complete cellular coverage.
Therefore, this likely lowers the measured numbers compared to if all cells were expressing the constructs.
Furthermore, due to highly clustered mRNAs in SGs, accurate quantification was proven to be rather difficult also decreasing the number.
Second, G3BP1 has a significant effect on protein expression irrespective of cellular stress.
This interesting finding showed that the role of G3BP1 in SGs might just be secondary while its more important function occurs in normal cell homeostasis.
As previously proposed, SGs might arise from the necessity to sequester proteins like G3BP1 from mRNA targets in the cytosol \cite{fischer_structure-mediated_2020}.
SunTag imaging experiments were performed to observe differences in translational activity.
These, however, did not show major differences suggesting that G3BP1 might only have minor roles in translational activity.
Interestingly, TREAT assays looking at mRNA stability showed significant differences.
This suggests that G3BP1 tethering reduces 5$'$-3$'$ mRNA degradation.
While some recent publications have provided evidence for a decay promoting role of G3BP1 \cite{fischer_structure-mediated_2020, tourriere_rasgap-associated_2001}, others including this study have suggested the opposite \cite{aulas_g3bp1_2015, bley_stress_2015, laver_rna-binding_2020}.
It must be noted, that this mRNA stabilizing effect, while significant, still does not prevent mRNAs from degrading entirely.
The molecular mechanisms behind this finding remain unknown but could arise from a microscopic SG-like environment forming around single mRNAs.
Even though G3BP1 tethering does not form visible SGs in unstressed cells, the recruitment of SG proteins by G3BP1 could still be sufficient to reduce mRNA degradation.
Similarly, G3BP1 could also actively be preventing the recruitment of decay enzymes to their tethered target mRNAs.
To follow up on these proposals, it would be necessary to investigate whether the local protein composition changes proximally to tethered mRNAs.
A similar experimental setup could also be used to test the mRNA stabilizing effect of other SG-proteins like TIA-1, single G3BP1 domains, or domains enriched in the SG transcriptome such as DEAD boxes.
Lastly, the direct involvement of G3BP1 on mRNA stabilization during stress conditions has not been investigated and must be addressed to understand the overall role of G3BP1 in mRNA metabolism.

In an effort to increase the expression levels of the currently used SunTag reporter cassette, a novel cassette relying on structural support by smGFP was created.
The preliminary results shown here suggest that single-chain antibody occupancies are comparable with the previous version while greatly increasing transcript numbers.
This seems to mainly be achieved through more translating transcripts per cell.
However, it is still unclear what the underlying cause of the poor induction of the SunTag reporter is.
The lower signal intensity has shown to make a track based analysis slightly harder.
To counteract this, one could, however, use two or more cassettes, or add more GCN4 repeats at other locations within the structural scaffold.
Looking forward, it would be interesting to see if mRNA stability is also affected and how the system can be evolved.
Recent progress in single-chain antibodies leads the way to a highly interchangeable system with different tags and fluorescent colors.
